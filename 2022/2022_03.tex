\begin{center}
\section*{\month}
\end{center}

\def\day{\textit{March 15th, 2022}}
\def\weekday{\textit{Tuesday}}
\subsection*{\weekday, \day}


제1장. 평범한 일을 일기에 쓴다
\begin{itemize}
  \item 집에서 한가롭게 지냈다\\
  
  하루 종일 집에서 청소를 하고 TV를 보고 신문을 읽으며 한가롭게 지냈다.\\
  
  I stayed home all day cleaning, watching TV and reading the paper.\\
  
  \item 클래식으로 위로받는 시간\\
  
  아침에 클래식 음악을 들었다. 아주 조용하고 편안한 시간이었다.\\
  
  I listened to classical music this morning. I had a very quiet and relaxing time.\\
  
  \item 피곤하니까 일직 자야지\\
  
  오늘은 할 일이 많아서 아주 피곤했다. 오늘 밤에는 일찍 자야겠다.\\
  
  I had a lot of things to do today, so I'm very tired. I'll go to bed early tonight.\\
  
  \item 기름이 오른 고등어\\
  
  오늘 아침에 슈퍼에서 산 고등어는 아주 신선하고 기름이 올라 맛이 있었다.\\
  
  The \underline{mackerel} that I bought at the grocery store this morning was very fresh, fatty and delicious.\\
  
  \item 시시한 쇼\\
  
  나는 제리 스프링어 쇼를 봤다. 시시한 건 알지만 아무래도 자꾸 보게 된다.\\
  
  I watched the \textit{Jerry Springer Show}. I know it's trash but sometimes I \underline{can't help watching} it.\\
  
  \item 보고서 정리\\
  
  나는 새벽 4시에 일어나서 보고서를 정리했다. 정오까지는 제출해야 하는데 가까스로 시간에 맞췄다.\\
  
  I got up at 4:00 a.m. and worked on the report. I had to turn it in by noon and I \underline{barely made it}.\\
  
  \item 보름달\\
  
  오늘 밤 달은 아주 크로 매우 아름다웠다. 별이 보이지 않을 정도로 밝았다.\\
  
  The moon tonight was huge and very beautiful. It was so bright that the star couldn't be seen.\\
  
  \item 초밥\\
  
  나는 식구들을 초밥집에 데려갔다. 아이들이 워낙 많이 먹었기 때문에 예상보다 돈을 많이 썼다.\\
  
  I brought my family to susi restaurant. I wasted more than I expected since children had a lot.\\
  
  I took my family to a \textit{sushi} bar. Our children really ate a lot and I ended up paying much more than I expected.\\
  
  \item 구인 광고\\
  
  나는 신문 구인 광고를 대충 훑어봤다. 새로운 모집 공고는 없었다. 내 흥미를 끄는 회사가 아직 없다. 좀더 찾아봐야겠다.\\
  
  I skimmed \textcolor{red}{through} \sout{newspaper recruitment section} employment ads. There was not a new recruitment. \\
  
  I scanned through the \underline{help-wanted ads} in the paper. Nothing new. There still isn't a company that interests me. I guess I'll keep looking.\\
  
  \item 피자 주문\\
  
  진수와 하나가 놀러왔다. 우리는 배가 너무 고파서 빅 파파에서 피자를 시켜서 얘기하면서 먹었다.\\
  
  Jin-su and Ha-na came to see me. We were so hungry that we ordered pizza, and ate it while talking.\\
  
  Jin-su and Ha-na came to see me. We were all very hungry, so we got some pizzas from Big Papa's and ate them while talking.\\
  
\pagebreak
\def\day{\textit{March 16th, 2022}}
\def\weekday{\textit{Wednesday}}
\subsection*{\weekday, \day}
  
  \item 심부름\\
  
  나는 엄마 심부름을 했다. 우체국에 가서 우표를 사고 도서관에서 책을 반납한 뒤 엄마 재킷을 세탁소에 맡겼다. 무척 바빴다.\\
  
  I \underline{ran some errands} for Mom. I went to the post-office to get some stamps, returned some books to the library and \underline{took her jacket to the dry cleaner's}. I was quite busy.\\
  
  \item 잡초 뽑기\\
  
  마당에 잡초가 우거져서 잡초를 뽑았다. 잡초를 전부 뽑는 데 2시간이 걸렸다. 깨끗해져서 기분은 좋지만 무릎이 아프다.\\
  
  I removed weeds on the yards. It took 2 hours to remove all. It feels good to remove the weeds, but my knees feel pain.\\
  
  I \underline{weeded} my garden since it was completely overgrown with weeds. It took me two hours to get all the weeds out. I'm glad it looks much better but my knees are \underline{sore}.\\
  
  \item 기타 연습\\
  
  나는 몇 시간 동안 기타를 연습했다. 막 배우기 시작했기 때문에 아직 잘 치지는 못한다. 하지만 다소 발전하고 있다는 느낌이 들어 기쁘다.\\
  
  I practiced guitar for several hours. Since I started just now for learning, I am not good at it. But I feel better and better. It makes me pleased.\\
  
  I practiced my guitar for a few hours. I just started talking lessons and \underline{I'm not any good yet}. But \underline{I'm pleased with myself} when I see some improvement.\\
  
  \item 새로운 타입의 음악\\
  
  라디오에서 아주 멋진 음악이 흘러나왔다. 펑크 같은데 태평소와 아쟁으로 연주했다. 지금까지 들어본 적이 없는 새로운 타입의 음악으로 상당히 마음에 들었다.\\
  
  Very fancy music comes out of the radio. It sounds like funk ...\\
  
  I heard a very cool song on the radio. It was sort of like punk but played using \textit{Taepyeongso} and \textit{Ajaeng}. It sounded like a new type of music that I'd never heard before, and I love \\
  
  \item 피아노 레슨\\
  
  딸아이가 피아노에 흥미를 보이기 시작해서 시험 레슨에 데려갔다. 아이는 즐거운 듯이 건반을 두드려봤다. 선생님도 마음에 드는 모양이었다. 딸아이는 교실을 나올 때 이미 레슨을 받기로 결정했다고 말했다.\\
  
  I took my daughter to a piano lesson because she is interested in piano. \textcolor{red}{She had fun playing, and \underline{she liked the instructor.}} \sout{Teacher liked her.} She said she made her mind to get lesson when she comes out of the classroom.\\
  
  My daughter showed some interest in the piano, so I took her to a trial lesson. She had fun playing, and she liked the instructor. As she was leaving the class, she told me that she had already decided to start taking lessons.\\
  
  \item 녹차 쉬폰케이크\\
  
  오늘 현정 씨가 녹차 시폰 케이크를 구워서 영어 교실에 가져왔다. 정말 맛있었다! 그녀는 만들기 쉽다며 우리에게 요리법을 알려줬다. 시간이 나면 나도 만들어봐야겠다.\\
  
  Hyeonjung took a green-tea-chiffon-cake to the English class. It was delicious! She told us the recipe saying it's easy. I would try when I have time.\\
  
  Hyeon-jeong baked a green-tea-flavored chiffon cake and brought it to our English class today. It was very yummy! She said it was easy to make and gave us the recipe. I'll try it myself when I get a chance.\\
  
  \item 팝콘을 받아먹는 갈매기\\
  
  점심을 먹은 후 리사, 진저, 나 셋이서 근처 바닷가에 갔다. 아주 많은 갈매기들이 젊은 남자가 던져주는 팝콘을 먹으려고 모여들었다. 가끔씩 팝콘 한 알 때문에 싸우기도 했다. 아주 흥미로운 광경이었다.\\
  
  After lunch, Lisa, Jinger and I went to a nearby beach. A lot of seagulls gathered to eat popcorns that a young guy throws. Sometimes, they fight for a piece of pop-corn. It was very interesting scene.\\
  
  Lisa, Ginger and I went to the beach nearby after lunch. Countless seagulls were catching popcorn that a young man was throwing. Sometimes they fought over a piece of popcorn. It was an interesting scene.\\
  
\end{itemize}